%%%%%%%%%%%%%%%%%%%%%%%%%%%%%%%%%%%%%%%%%
% Friggeri Resume/CV
% XeLaTeX Template
% Version 1.2 (3/5/15) 
%
% This template has been downloaded from:
% https://github.com/mlda065/friggeri-letter
%
% Original author: 
% Adrien Friggeri (adrien@friggeri.net)
% https://github.com/afriggeri/CV
%
% Modifications by Matthew Davis (matthew@mdavis.xyz)
%
% License:
% CC BY-NC-SA 3.0 (http://creativecommons.org/licenses/by-nc-sa/3.0/)
% 
% Important notes:
% This template needs to be compiled with XeLaTeX 
%
%%%%%%%%%%%%%%%%%%%%%%%%%%%%%%%%%%%%%%%%%

\documentclass[]{friggeri-cv} % Add 'print' as an option into the square bracket to remove colors from this template for printing
 

\begin{document} 
 
\header{Frander}{Granados} % Your name and current job title/field 

%----------------------------------------------------------------------------------------
%	SIDEBAR SECTION 
%----------------------------------------------------------------------------------------

 
\aside{
\subsubsection{contacto}
%don't display the name in the contact details on the first page
%because it's up the top%That's why there's the if statement
\ifx \firstPage \undefined
   \xdef\firstPage{}
\else
Frander Granados \\
~ \\
\fi
Santiago\\
Puriscal, San José 40101 \\
Costa Rica \\
~  \\
+(506) 8663 3191 \\
+(506) 2416 7391 \\
~   \\
\href{mailto:fgranadosvega@gmail.com}{fgranadosvega \\ @gmail.com} \\
%\href{http://www.smith.com}{http://www.smith.com}
~ \\
~ \\
\subsubsection{idiomas}
Español lenguaje materno \\
Inglés medio
} 


%----------------------------------------------------------------------------------------
% EXPERIENCE SECTION
%----------------------------------------------------------------------------------------

\section{Experiencia}
 
\begin{entrylist}

%------------------------------------------------

\entry 
{2013--Ahora}
{Instituto Tecnológico de Costa Rica}
{Cartago, Costa Rica}%side thing
{Soporte técnico}
{
Trabajo como asistente en el área de soporte de la Escuela de Ingeniería en Computación, parte de mi trabajo es la creación y mantenimiento de servidores GNU/Linux y Microsoft Windows, revisión del equipo de los profesores y laboratorios de la Escuela.
}

%------------------------------------------------

%\end{entrylist}
%
%\subsection{Part Time}
%
%\begin{entrylist}
%
%
%\entry
%{2013-2015}
%{Company} 
%{side thing}
%{Job title}
%{ 
%Description
%}
%
%
%%------------------------------------------------
%
%
%\end{entrylist}
%
%
%%------------------------------------------------
%
%
%\subsection{Community Roles}
%
%\begin{entrylist}
%
%
%\entry
%{2012-Now}
%{Organisation name} 
%{}
%{Role}
%{
%description
%}
%
%
%%------------------------------------------------


\end{entrylist}
%----------------------------------------------------------------------------------------
%	EDUCATION SECTION
%----------------------------------------------------------------------------------------


\section{Educación}

\begin{entrylist} 

%------------------------------------------------

\entry
{2006--2010}
{Bachillerato {\normalfont Educación secundaria}}
{Liceo de Puriscal}
{
}
%------------------------------------------------

\entry
{2011-- ahora}  
{Lic. Ingeniería en Computadores}
{Instituto Tecnológico de Costa Rica}
{
}
%------------------------------------------------

\end{entrylist}

%----------------------------------------------------------------------------------------
%	AWARDS SECTION
%----------------------------------------------------------------------------------------

\section{Otros cursos}

\begin{entrylist}

%------------------------------------------------

\entry
{2016}
{Física de Plasmas y sus aplicaciones}
{Instituto Tecnológico de Costa Rica} 
{} 
{}

%------------------------------------------------

\entry
{2017}
{Bases de datos Avanzadas}
{Instituto Tecnológico de Costa Rica} 
{} 
{}

%------------------------------------------------
\entry
{2018}
{Introducción a los sistemas embebidos}
{Instituto Tecnológico de Costa Rica} 
{} 
{}

%------------------------------------------------

\end{entrylist}

%----------------------------------------------------------------------------------------
%	Human SKILLS SECTION
%----------------------------------------------------------------------------------------

\section{Habilidades}
  \vspace{-0.2cm}

Empatía, Positivismo, Trabajo en equipo, Creatividad, Solución de problemas, Adaptabilidad.


%----------------------------------------------------------------------------------------
%	Technical SKILLS SECTION
%----------------------------------------------------------------------------------------

\section{Habilidades técnicas}

\begin{entrylist}

%------------------------------------------------

\entry
{}
{Programación}
{}
{}
{
C/C++, Python, \LaTeX, Java, Arduino, Verilog, SQL, NoSQL, HTML, bash

}

%------------------------------------------------

\entry
{}
{software y otros}
{}
{}
{
GNU/Linux, LTSpice, git, Raspberry Pi, FPGA, ARM, MIPS, RISC-V, circuitos eléctricos

}

%------------------------------------------------

\end{entrylist}

%----------------------------------------------------------------------------------------
%	INTERESTS SECTION
%----------------------------------------------------------------------------------------

\section{Intereses}
  \vspace{-0.2cm}

\textbf{profesional:} Sistemas Embebidos, Sitemas Operativos, Lenguajes de Programación, 
GNU/Linux, Arquitectura y Organización de Computadores, Internet of Things, Plasma, Circuitos eléctricos. \\
\textbf{personal:}  Cine, Música, Deportes, Fútbol, Viajar, Ciencia Ficción.

%----------------------------------------------------------------------------------------

\newpage

%----------------------------------------------------------------------------------------
%	References SECTION 
%----------------------------------------------------------------------------------------

\section{Referencias}


\begin{entrylist}

\entry
{}
{José Ángel Stradi Granados}
{M.Sc en Ingeniería en Computación}
{Profesor en el Instituto Tecnológico de Costa Rica}
{Teléfono:(+506) 8371 4083 \\
correo electrónico: jastradi@gmail.com
}


\entry
{}
{Jorge Castro Godínez}
{M.Sc. en Ingeniería Electrónica}
{Investigador en el Chain of Embedded Systems (CES) y estudiante de doctorado en el Instituto Tecnológico de Karlsruhe (KIT)}
{Teléfono: +49 721 608 46050 \\
correo electrónico: jorge.castro-godinez@kit.edu
}

\end{entrylist} 

%----------------------------------------------------------------------------------------

\end{document}